\chapter[%
\bibname{}][%
\bibname{}]{%
\bibname{}}
\label{chap:Bibliographie}
\bibmark{}

Les références bibliographiques sont classées par ordre alphabétique du nom
des auteurs puis par ordre chronologique de la date de publication, selon le
schéma \Quote{auteurs/date}.
Les pages du document qui mentionnent une référence sont indiquées à la fin
de l'entrée, entre parenthèses.
La translittération scientifique n'est pas appliquée aux titres originaux
des publications, qui doivent toujours rester intacts.
La traduction en français des titres originaux en russe, donnée entre
crochets, est approximative.
La liste des citations en page~\pageref{chap:Listedescitations} et l'index
onomastique en page~\pageref{chap:Indexonomastique} complètent, sans
répétitions, la bibliographie.

\Comment{Dans le cas de la revue \emph{Советская музыка} (translittération
scientifique \emph{Sovetskaja muzyka}), renommée \emph{Музыкальная академия}
(\emph{Muzykal'naja akademija}) au début de l'année~1992, le \Quote{volume}
désigne le numéro cumulatif indiqué sur le site Internet
\href{https://mus.academy/}{https://mus.academy/}, entre parenthèses à côté
de l'année.
\emph{Музыкальная академия} a le numéro ISSN~0869-4516.
Cette revue comporte une section intitulée \emph{Музыкальная жизнь}, à ne
pas confondre avec la revue du même nom (\emph{Muzykal'naja žizn'}), qui a
le numéro ISSN~0131-2383.}

\section{Subdivisions thématiques de la bibliographie}

\begin{description}
 \item[Ouvrages et articles russes principaux]
 Ouvrages collectifs en russe dirigés par \citet{Milshteyn70, Milshteyn82a},
 par \citet{Lobanov03}, par \citet{Nikonovich08} et par \citet{Scriabine}.
 L'énumération comporte en outre les chapitres ou articles de ces ouvrages
 collectifs.
 \item[Autres ouvrages et articles russes]
 Autres monographies, ouvrages, thèses, contributions, articles,
 publications en ligne,~etc.
 Toutes ces références sont rédigées en russe.
 \item[Sources et données discographiques]
 Discographies de \VSofronitsky{} établies par \citet{White}, \citet{Malik},
 \citet{Masuda}, \citet{Taylor}, \citet{Graham}, \citet{Nikonovich11},
 \citet{Rossi} et \citet{Johansson}.
 L'énumération comporte en outre les autres sources discographiques servant
 à l'établissement de ce document.
 \item[Monographies et articles généraux]
 Monographies, ouvrages, contributions, articles, publications en ligne,
 livrets d'accompagnement de disques vinyles et compacts, documentaires,
 émissions radiophoniques,~etc.
 \item[Thèses et articles de recherche]
 Thèses de doctorat ou de maîtrise, articles de recherche, contributions aux
 actes de conférences,~etc.
 \item[Travaux lexicographiques et dictionnaires]
 Entrées dans des encyclopédies, des dictionnaires et des lexiques.
 Il ne s'agit que de références brèves et synthétiques.
 \item[Critiques de disques vinyles et compacts]
 Critiques et recensions de disques vinyles et compacts parues dans la
 presse spécialisée.
 \item[Réalisation et composition de la discographie]
 Outils logiciels servant à la réalisation de la discographie puis à sa
 composition~: système d'exploitation et utilitaires, programmes permettant
 l'écoute et la comparaison des fichiers audio, système de composition
 typographique, polices de caractères, licence utilisée,~etc.
\end{description}

\printbibliography[%
    heading=subbibnumbered,%
    title={Ouvrages et articles russes principaux},%
    keyword=primarysource,%
]
\printbibliography[%
    heading=subbibnumbered,%
    title={Autres ouvrages et articles russes},%
    keyword=othersource,%
]
\printbibliography[%
    heading=subbibnumbered,%
    title={Sources et données discographiques},%
    keyword=discography,%
]
\printbibliography[%
    heading=subbibnumbered,%
    title={Monographies et articles généraux},%
    keyword=bookarticle,%
]
\printbibliography[%
    heading=subbibnumbered,%
    title={Thèses et articles de recherche},%
    keyword=research,%
]
\printbibliography[%
    heading=subbibnumbered,%
    title={Travaux lexicographiques et dictionnaires},%
    keyword=lexicography,%
]
\printbibliography[%
    heading=subbibnumbered,%
    title={Critiques de disques vinyles et compacts},%
    keyword=discrecension,%
]
\printbibliography[%
    heading=subbibnumbered,%
    title={Réalisation et composition de la discographie},%
    keyword=comparetypeset,%
]
