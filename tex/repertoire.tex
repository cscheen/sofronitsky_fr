\chapter[%
Répertoire de Vladimir Sofronickij][%
Répertoire de Vladimir Sofronickij]{%
Répertoire de \VSofronitsky{}}
\label{chap:Repertoire}

\section{Références bibliographiques}

Le répertoire pianistique de \VSofronitsky{} -- œuvres pour piano solo,
œuvres pour piano et orchestre, œuvres de musique de chambre -- était
beaucoup plus étendu que ce qui nous en est parvenu sous la forme
d'enregistrements en studio ou en direct et sous celle de bandes
radiophoniques.
Selon \citet[p.~462-465]{Milshteyn82a}, \citet[p.~72-82]{White},
\citet{Nekrasova08}, \citet{Badeyan10a} et \citet[p.~40-58]{Lajko}, les
œuvres et compositeurs suivants figuraient au répertoire du pianiste.
L'énumération n'est cependant pas exhaustive, du moins en ce qui concerne
\citet{Nekrasova08} et surtout \citet{Badeyan10a}, mais elle permet de mieux
appréhender l'ampleur et la diversité du répertoire pianistique de
\VSofronitsky{}, au-delà de sa discographie connue à ce jour.
Les informations reprises dans la chronologie des récitals éditée par
\citet[p.~393-452]{Scriabine} ont permis de compléter ce chapitre.
Le classement des compositeurs est chronologique sur la base de l'année de
naissance.
Pour certains compositeurs, le répertoire est encore subdivisé en fonction
du genre musical.

\section[%
Dietrich Buxtehude (1637-1707) -- Leonid Nikolaev]{%
\DBuxtehude{} (1637-1707) -- \LNikolaiev{}}

Prélude et fugue pour orgue en \kF \Sharp mineur.

\section[%
Johann Pachelbel (1653-1706) -- Leonid Nikolaev]{%
\JPachelbel{} (1653-1706) -- \LNikolaiev{}}

\emph{Toccata} pour orgue en \kF majeur.

\section[%
Johann Sebastian Bach (1685-1750)]{%
\JBach{} (1685-1750)}

\subsection{Art de la fugue}

Deux Fugues extraites de l'\hbox{Art} de la fugue, BWV~1080.

\subsection{Clavier bien tempéré}

Douze Préludes et fugues extraits du Clavier bien tempéré, BWV~846-893, y
compris ceux en \kG majeur, en \kG mineur, en \kB \Flat majeur, en \kB \Flat
mineur et d'autres.

\subsection{Concertos}

Concertos pour deux claviers et cordes en \kC mineur et en \kC majeur.

\emph{Aria}.
Il s'agit peut-être du \emph{Largo} du Concerto \Number{5} en \kF mineur,
BWV~1056, selon \citet[p.~72, note~2]{White}.

\subsection{Suites anglaises et Suites françaises}

Suites anglaises \Number{2} en \kA mineur, BWV~807, et \Number{3} en \kG
mineur, BWV~808.

Suites françaises \Number{1} en \kD mineur, BWV~812, \Number{5} en \kG
majeur, BWV~816, et \Number{6} en \kE majeur, BWV~817.

\subsection{Autres œuvres}

\emph{Rondo}.

\section[%
Johann Sebastian Bach (1685-1750) -- Ferruccio Busoni]{%
\JBach{} (1685-1750) -- \FBusoni{}}

Dix Préludes de chorals, y compris ceux en \kG mineur, en \kE \Flat mineur,
en \kF mineur, en \kC majeur, en \kG majeur, en \kA mineur, en \kD mineur et
d'autres.

Toccatas et fugues pour orgue en \kC majeur, en \kD mineur et en \kE mineur.

\section[%
Johann Sebastian Bach (1685-1750) -- Ferenc Liszt]{%
\JBach{} (1685-1750) -- \FLiszt{}}

Préludes et fugues pour orgue en \kA mineur et en \kD majeur.

\section[%
Johann Sebastian Bach (1685-1750) -- Aleksandr Ziloti]{%
\JBach{} (1685-1750) -- \AZiloti{}}

Préludes en \kE mineur et en \kB mineur.

\section[%
Georg Friedrich Haendel (1685-1759)]{%
\GHaendel{} (1685-1759)}

\emph{Aria} et variations en \kE majeur, mouvement final de la Suite
\Number{5} en \kE majeur, HWV~430, \Quote{L'harmonieux forgeron}.

Variations en \kB \Flat majeur.

\section[%
Domenico Scarlatti (1695-1757)]{%
\DScarlatti{} (1695-1757)}

Douze Sonates, y compris celles en \kA majeur, en \kB mineur, en \kG majeur,
en \kE mineur, en \kD majeur et d'autres.

\section[%
Carl Philipp Emanuel Bach (1714-1788)]{%
\CBach{} (1714-1788)}

\emph{Rondo espressivo}.
Le plus probable est qu'il s'agisse du \emph{Rondo} à deux voix en \kB
mineur extrait de la Première Collection \emph{Für Kenner und Liebhaber},
Wq.~55, selon \citet[p.~72, note~3]{White}.

\section[%
Joseph Haydn (1732-1809)]{%
\JHaydn{} (1732-1809)}

Sonates \Number{53} en \kE mineur, Hob.XVI:34, et \Number{50} en \kD majeur,
Hob.XVI:37.

Variations en \kF mineur, Hob.XVII:6, \Quote{\emph{Un piccolo
divertimento}}.

\section[%
Dmitrij Bortnjanskij (1751-1825)]{%
\DBortnianski{} (1751-1825)}

Sonate en \kC majeur (1784).

\section[%
Muzio Clementi (1752-1832)]{%
\MClementi{} (1752-1832)}

Finale d'une Sonate.

\section[%
Wolfgang Amadeus Mozart (1756-1791)]{%
\WMozart{} (1756-1791)}

\subsection{Fantaisies}

Fantaisies \Number{1} avec fugue en \kC majeur, K~394, \Number{2} en \kC
mineur, K~396/385f, \Number{3} en \kD mineur, K~397, et \Number{4} en \kC
mineur, K~475.

\subsection{Sonates}

Sonates \Number{3} en \kB \Flat majeur, K~281, \Number{4} en \kE \Flat
majeur, K~282, \Number{11} en \kA majeur, K~331, \Number{14} en \kC mineur,
K~457, et \Number{18} en \kD majeur, K~576.

\subsection{Autres œuvres}

\emph{Rondo} \Number{3} en \kA mineur, K~511.

Sonate pour deux pianos en \kD majeur, K~448.

Fugue.

\section[%
Wolfgang Amadeus Mozart (1756-1791) -- Ferruccio Busoni]{%
\WMozart{} (1756-1791) -- \FBusoni{}}

Fantaisie pour orgue en \kF mineur, pour deux pianos.

\section[%
Ludwig van Beethoven (1770-1827)]{%
\LBeethoven{} (1770-1827)}

\subsection{Rondos}

\emph{Rondo} en \kC majeur, \Opus{51} \Number{1}, et \emph{Rondo} en \kG
majeur, \Opus{51} \Number{2}.

\subsection{Sonates}

Sonates \Number{1} en \kF mineur, \Opus{2} \Number{1}, \Number{2} en \kA
majeur, \Opus{2} \Number{2}, \Number{3} en \kC majeur, \Opus{2} \Number{3},
\Number{5} en \kC mineur, \Opus{10} \Number{1}, \Number{6} en \kF majeur,
\Opus{10} \Number{2}, \Number{7} en \kD majeur, \Opus{10} \Number{3},
\Number{8} en \kC mineur, \Opus{13}, \Number{9} en \kE majeur, \Opus{14}
\Number{1}, \Number{12} en \kA \Flat majeur, \Opus{26}, \Number{14} en \kC
\Sharp mineur, \Opus{27} \Number{2}, \Number{15} en \kD majeur, \Opus{28},
\Number{23} en \kF mineur, \Opus{57}, \Number{24} en \kF \Sharp majeur,
\Opus{78}, \Number{25} en \kG majeur, \Opus{79}, \Number{26} en \kE \Flat
majeur, \Opus{81a}, \Number{27} en \kE mineur, \Opus{90}, et \Number{32} en
\kC mineur, \Opus{111}.

\subsection{Variations}

Variations en \kG majeur, en \kD majeur (sans doute \Opus{76}~: Six
Variations sur le thème original de la Marche turque extraite des Ruines
d'\hbox{Athènes}) et en \kA majeur.

Trente-deux Variations en \kC mineur, WoO~80.

\subsection{Autres œuvres}

Bagatelles extraites de l'\Opus{33}.

\emph{Andante} favori en \kF majeur, WoO~57.

Polonaise en \kC majeur, \Opus{89}.

\section[%
Franz Schubert (1797-1828)]{%
\FSchubert{} (1797-1828)}

\subsection{Impromptus}

Quatre Impromptus, D~899 (\Opus{90})~: \Number{1} en \kC mineur, \Number{2}
en \kE \Flat majeur, \Number{3} en \kG (\Flat) majeur et \Number{4} en \kA
\Flat majeur.

Deux Impromptus extraits du recueil D~935 (\Opus{142})~: \Number{1} en \kF
mineur et \Number{2} en \kA \Flat majeur.

\subsection{Moments musicaux}

Six Moments musicaux, D~780 (\Opus{94})~: \Number{1} en \kC majeur,
\Number{2} en \kA \Flat majeur, \Number{3} en \kF mineur, \Number{4} en \kC
\Sharp mineur, \Number{5} en \kF mineur et \Number{6} en \kA \Flat majeur.

\subsection{Sonates}

Sonates en \kA majeur, D~664 (\Opus{120}), en \kA mineur, D~784
(\Opus{143}), en \kA mineur, D~845 (\Opus{42}), et en \kB \Flat majeur,
D~960 (\Opus{posthume}).

\subsection{Autres œuvres}

Fantaisie \Quote{\emph{Wanderer}} en \kC majeur, D~760 (\Opus{15}).

Trois Sonatines pour piano et violon~: en \kD majeur, D~384 (\Opus{137}
\Number{1}), en \kA mineur, D~385 (\Opus{137} \Number{2}), et en \kG mineur,
D~408 (\Opus{137} \Number{3}).

Trio \Number{1} pour piano, violon et violoncelle en \kB \Flat majeur, D~898
(\Opus{99}).

\section[%
Franz Schubert (1797-1828) -- Ferenc Liszt]{%
\FSchubert{} (1797-1828) -- \FLiszt{}}

\subsection{Transcriptions de lieder}

\emph{Aufenthalt}, D~957 \Number{5} (S~560 \Number{3})~; \emph{Ihr Bild},
D~957 \Number{9} (S~560 \Number{8})~; \emph{Der Doppelgänger}, D~957
\Number{13} (S~560 \Number{12})~; \emph{Erlkönig}, D~328 (S~558
\Number{4})~; \emph{Erstarrung}, D~911 \Number{4} (S~561 \Number{5})~;
\emph{Die Forelle}, D~550 (S~563 \Number{6})~; \emph{Frühlingsglaube}, D~686
(S~558 \Number{7})~; \emph{Gretchen am Spinnrade}, D~118 (S~558
\Number{8})~; \emph{Litanei}, D~343 (S~562 \Number{1})~; \emph{Am Meer},
D~957 \Number{12} (S~560 \Number{4})~; \emph{Der Müller und der Bach}, D~795
\Number{19} (S~565 \Number{2})~; \emph{Die junge Nonne}, D~828 (S~558
\Number{6})~; \emph{Die Post}, D~911 \Number{13} (D~561 \Number{4})~;
\emph{Du bist die Ruh}, D~776 (S~558 \Number{3})~; \emph{Die Stadt}, D~957
\Number{11} (S~560 \Number{1})~; \emph{Ständchen} (\Quote{\emph{Horch,
horch~! die Lerch}}), D~889 (S~558 \Number{9})~; \emph{Auf dem Wasser zu
singen}, D~774 (S~558 \Number{2})~; \emph{Wohin~?}, D~795 \Number{2} (S~565
\Number{5}).

\subsection{Valses-caprices}

Six Valses-caprices (y compris l'\emph{Allegro spiritoso}, \Number{7} en \kA
majeur, S~427 \Number{7}) extraites des Soirées de Vienne.

\section[%
Franz Schubert (1797-1828) -- Sergej Prokof'ev]{%
\FSchubert{} (1797-1828) -- \SProkofiev{}}

Suite de valses pour deux pianos.

\section[%
Michail Glinka (1804-1857)]{%
\MGlinka{} (1804-1857)}

Nocturne.

Variations sur un thème écossais (1847).

\section[%
Felix Mendelssohn-Bartholdy (1809-1847)]{%
\FMendelssohn{} (1809-1847)}

Études en \kF majeur, \Opus{104b} \Number{2}, et en \kA mineur, \Opus{104b}
\Number{3}.

Douze pièces extraites des \emph{Lieder ohne Worte}.

Prélude.

Variations sérieuses en \kD mineur, \Opus{54}.

\section[%
Fryderyk Chopin (1810-1849)]{%
\FChopin{} (1810-1849)}

\subsection{Ballades}

Quatre Ballades~: \Number{1} en \kG mineur, \Opus{23}, \Number{2} en \kF
majeur, \Opus{38}, \Number{3} en \kA \Flat majeur, \Opus{47}, et \Number{4}
en \kF mineur, \Opus{52}.

\subsection{Études}

Seize Études extraites des \Opus{10 et~25}.
Dans l'\Opus{10}~: \Number{2} en \kA mineur, \Number{3} en \kE majeur,
\Number{4} en \kC \Sharp mineur, \Number{5} en \kG \Flat majeur, \Number{6}
en \kE \Flat mineur, \Number{8} en \kF majeur, \Number{9} en \kF mineur et
d'autres (dont une en \kC majeur).
Dans l'\Opus{25}~: \Number{1} en \kA \Flat majeur, \Number{2} en \kF mineur,
\Number{3} en \kF majeur, \Number{4} en \kA mineur, \Number{5} en \kE
mineur, \Number{7} en \kC \Sharp mineur, \Number{12} en \kC mineur et
d'autres.

Trois Nouvelles Études pour la Méthode des méthodes de \Moscheles{} et
\Fetis{}~: \Number{1} en \kF mineur, \Number{2} en \kA \Flat majeur et
\Number{3} en \kD \Flat majeur.

\subsection{Impromptus}

Quatre Impromptus~: \Number{1} en \kA \Flat majeur, \Opus{29}, \Number{2} en
\kF \Sharp majeur, \Opus{36}, \Number{3} en \kG \Flat majeur, \Opus{51}, et
\Number{4} (Fantaisie-impromptu) en \kC \Sharp mineur, \Opus{66}.

\subsection{Mazurkas}

Vingt-quatre Mazurkas~: \Opus{6} \Number{3} en \kE majeur, \Opus{6}
\Number{4} en \kE \Flat mineur, \Opus{7} \Number{3} en \kF mineur, \Opus{17}
\Number{1} en \kB \Flat majeur, \Opus{17} \Number{3} en \kA \Flat majeur,
\Opus{24} \Number{1} en \kG mineur, \Opus{30} \Number{2} en \kB mineur,
\Opus{30} \Number{3} en \kD \Flat majeur, \Opus{30} \Number{4} en \kC \Sharp
mineur, \Opus{33} \Number{1} en \kG \Sharp mineur, \Opus{33} \Number{2} en
\kD majeur, \Opus{33} \Number{3} en \kC majeur, \Opus{33} \Number{4} en \kB
mineur, \Opus{41} \Number{1} en \kC \Sharp mineur, \Opus{41} \Number{2} en
\kE mineur, \Opus{41} \Number{4} en \kA \Flat majeur, \Opus{50} \Number{3}
en \kC \Sharp mineur, \Opus{63} \Number{1} en \kB majeur, \Opus{63}
\Number{2} en \kF mineur, \Opus{67} \Number{4} en \kA mineur, \Opus{68}
\Number{1} en \kC majeur, \Opus{68} \Number{2} en \kA mineur, \Opus{68}
\Number{3} en \kF majeur et \Opus{68} \Number{4} en \kF mineur.

\subsection{Nocturnes}

Quatorze Nocturnes~: \Opus{9} \Number{2} en \kE \Flat majeur, \Opus{9}
\Number{3} en \kB majeur, \Opus{15} \Number{1} en \kF majeur, \Opus{15}
\Number{2} en \kF \Sharp majeur, \Opus{15} \Number{3} en \kG mineur,
\Opus{27} \Number{1} en \kC \Sharp mineur, \Opus{27} \Number{2} en \kD \Flat
majeur, \Opus{32} \Number{1} en \kB majeur, \Opus{37} \Number{1} en \kG
mineur, \Opus{37} \Number{2} en \kG majeur, \Opus{48} \Number{1} en \kC
mineur, \Opus{48} \Number{2} en \kF \Sharp mineur, \Opus{55} \Number{1} en
\kF mineur et \Opus{62} \Number{2} en \kE majeur.

\subsection{Polonaises}

Six Polonaises~: \Opus{26} \Number{1} en \kC \Sharp mineur, \Opus{26}
\Number{2} en \kE \Flat mineur, \Opus{40} \Number{1} en \kA majeur,
\Opus{40} \Number{2} en \kC mineur, \Opus{44} en \kF \Sharp mineur et
\Opus{53} en \kA \Flat majeur.

\subsection{Préludes}

Vingt-quatre Préludes, \Opus{28}~: \Number{1} en \kC majeur~; \Number{2} en
\kA mineur~; \Number{3} en \kG majeur~; \Number{4} en \kE mineur~;
\Number{5} en \kD majeur~; \Number{6} en \kB mineur~; \Number{7} en \kA
majeur~; \Number{8} en \kF \Sharp mineur~; \Number{9} en \kE majeur~;
\Number{10} en \kC \Sharp mineur~; \Number{11} en \kB majeur~; \Number{12}
en \kG \Sharp mineur~; \Number{13} en \kF \Sharp majeur~; \Number{14} en \kE
\Flat mineur~; \Number{15} en \kD \Flat majeur~; \Number{16} en \kB \Flat
mineur~; \Number{17} en \kA \Flat majeur~; \Number{18} en \kF mineur~;
\Number{19} en \kE \Flat majeur~; \Number{20} en \kC mineur~; \Number{21} en
\kB \Flat majeur~; \Number{22} en \kG mineur~; \Number{23} en \kF majeur~;
\Number{24} en \kD mineur.

Prélude en \kC \Sharp mineur, \Opus{45}.

\subsection{Scherzos}

Quatre Scherzos~: \Number{1} en \kB mineur, \Opus{20}, \Number{2} en \kB
\Flat mineur, \Opus{31}, \Number{3} en \kC \Sharp mineur, \Opus{39}, et
\Number{4} en \kE majeur, \Opus{54}.

\subsection{Sonates}

Deux Sonates~: \Number{2} en \kB \Flat mineur, \Opus{35}, et \Number{3} en
\kB mineur, \Opus{58}.

\subsection{Valses}

Onze Valses~: \Opus{34} \Number{1} en \kA \Flat majeur, \Opus{34} \Number{2}
en \kA mineur, \Opus{42} en \kA \Flat majeur, \Opus{64} \Number{1} en \kD
\Flat majeur, \Opus{64} \Number{2} en \kC \Sharp mineur, \Opus{64}
\Number{3} en \kA \Flat majeur, \Opus{69} \Number{1} en \kA \Flat majeur,
\Opus{69} \Number{2} en \kB mineur, \Opus{70} \Number{1} en \kG \Flat
majeur, \Opus{70} \Number{2} en \kF mineur et \Opus{70} \Number{3} en \kD
\Flat majeur.

\subsection{Autres œuvres}

\emph{Allegro} de concert en \kA majeur, \Opus{46}.

Barcarolle en \kF \Sharp majeur, \Opus{60}.

Berceuse en \kD \Flat majeur, \Opus{57}.

Boléro en \kC majeur/\kA majeur, \Opus{19}.

Fantaisie en \kF mineur, \Opus{49}.

Tarentelle en \kA \Flat majeur, \Opus{43}.

Variations brillantes en \kB \Flat majeur, \Opus{12}.

\section[%
Robert Schumann (1810-1856)]{%
\RSchumann{} (1810-1856)}

\subsection{Sonates}

Trois Sonates~: en \kF \Sharp mineur, \Opus{11}, en \kF mineur, \Opus{14}
(\Quote{Concert sans orchestre}), et en \kG mineur, \Opus{22}.

\subsection{Autres œuvres}

Papillons, \Opus{2}.

Au moins un \emph{Intermezzo} extrait de l'\Opus{4}.

Toccata en \kC majeur, \Opus{7}.

\emph{Allegro} en \kB mineur, \Opus{8}.

Carnaval, \Opus{9} (\Quote{Scènes mignonnes composées pour le Pianoforte sur
quatre notes}).

Pièces extraites des \emph{Phantasiestücke}, \Opus{12}~: \emph{Des Abends}
en \kD \Flat majeur (\Number{1}), \emph{Aufschwung} en \kF mineur
(\Number{2}), \emph{Traumes Wirren} en \kF majeur (\Number{7}) et d'autres.

Études symphoniques, \Opus{13} et \Opus{posthume}.

\emph{Kinderszenen}, \Opus{15}.

\emph{Kreisleriana}, \Opus{16}.

Fantaisie en \kC majeur, \Opus{17}.

Arabesque en \kC majeur, \Opus{18}.

\emph{Humoreske} en \kB \Flat majeur, \Opus{20}.

Trois \emph{Novelettes} extraites de l'\Opus{21}~: \Number{1} en \kF majeur,
\Number{7} en \kE majeur et \Number{8} en \kF \Sharp mineur.

\emph{Faschingsschwank aus Wien}, \Opus{26}.

Trois Romances, \Opus{28}~: \Number{1} en \kB \Flat mineur, \Number{2} en
\kF \Sharp majeur et \Number{3} en \kB majeur.

Deux pièces extraites des \emph{Klavierstücke}, \Opus{32}~: Scherzo en \kB
\Flat majeur (\Number{1}) et Romance en \kD mineur (\Number{3}).

\emph{Andante} et variations en \kB \Flat majeur pour deux pianos,
\Opus{46}.

\emph{Bunte Blätter}, \Opus{99} (au moins les huit premiers numéros).

Trois \emph{Phantasiestücke}, \Opus{111}~: \Number{1} en \kC mineur,
\Number{2} en \kA \Flat majeur et \Number{3} en \kC mineur.

\emph{Albumblätter}, \Opus{124} (au moins huit numéros).

\section[%
Ferenc Liszt (1811-1886)]{%
\FLiszt{} (1811-1886)}

\subsection{Années de pèlerinage}

\subsubsection{Première année~: Suisse}

Chapelle de Guillaume Tell, en \kC majeur, S~160 \Number{1}~; Au Lac de
Wallenstadt, en \kA \Flat majeur, S~160 \Number{2}~; Pastorale, en \kE
majeur, S~160 \Number{3}~; Au Bord d'une source, en \kA \Flat majeur, S~160
\Number{4}~; Églogue, en \kA \Flat majeur, S~160 \Number{7}~; Les Cloches de
Genève~: Nocturne, en \kB majeur, S~160 \Number{9}.

\subsubsection{Deuxième année~: Italie}

\emph{Sposalizio}, en \kE majeur, S~161 \Number{1}~; \emph{Il penseroso}, en
\kC \Sharp mineur, S~161 \Number{2}~; \emph{Canzonetta del Salvator Rosa},
en \kA majeur, S~161 \Number{3}~; \emph{Sonetto~47 del Petrarca}, en \kD
\Flat majeur, S~161 \Number{4}~; \emph{Sonetto~104 del Petrarca}, en \kE
majeur, S~161 \Number{5}~; \emph{Sonetto~123 del Petrarca}, en \kA \Flat
majeur, S~161 \Number{6}~; Après une lecture de Dante~: \emph{Fantasia quasi
Sonata}, en \kD mineur, S~161 \Number{7}.

\subsubsection{Deuxième année (supplément)~: \emph{Venezia e Napoli}}

\emph{Gondoliera}, en \kF \Sharp majeur, S~162 \Number{1}~; \emph{Canzone},
en \kE \Flat mineur, S~162 \Number{2}~; \emph{Tarantella}, en \kG mineur,
S~162 \Number{3}.

\subsection{Études}

\subsubsection{Études de concert et \emph{Konzertetüden}}

\emph{La leggierezza}, S~144 \Number{2}~; \emph{Waldesrauschen}, S~145
\Number{1}~; \emph{Gnomenreigen}, S~145 \Number{2}.

\subsubsection{Études d'exécution transcendante}

Feux follets, en \kB \Flat majeur, S~139 \Number{5}~; Étude, en \kF mineur
(\emph{Appassionata}), S~139 \Number{10}.

\subsubsection{Grandes Études de \Paganini{}}

Étude \Number{1}, en \kG mineur (Prélude, \emph{Andante}~; Étude --
\emph{Non troppo lento}), S~141 \Number{1}~; Étude \Number{2}, en \kE \Flat
majeur (\emph{Andante capriccioso}), S~141 \Number{2}~; Étude \Number{3}, en
\kG \Sharp mineur (\emph{Allegretto} -- \emph{La campanella}), S~141
\Number{3}~; Étude \Number{4}, en \kE majeur (\emph{Vivo}), S~141
\Number{4}~; Étude \Number{5}, en \kE majeur (La chasse), S~141 \Number{5}~;
Étude \Number{6}, en \kA mineur (Thème et variations), S~141 \Number{6}.

\subsection{Harmonies poétiques et religieuses}

\emph{Ave Maria}, S~173 \Number{2}~; Funérailles, S~173 \Number{7}~;
\emph{Andante lagrimoso}, S~173 \Number{9}.

\subsection{Rhapsodies hongroises}

Rhapsodie hongroise \Number{2} en \kC \Sharp mineur/\kF \Sharp majeur, S~244
\Number{2}~; Rhapsodie hongroise \Number{3} en \kB \Flat mineur/\kB \Flat
majeur, S~244 \Number{3}~; Rhapsodie hongroise \Number{6} en \kD \Flat
majeur/\kB \Flat majeur, S~244 \Number{6}~; Rhapsodie hongroise \Number{9}
en \kE \Flat majeur (\Quote{Carnaval de Pest}), S~244 \Number{9}~; Rhapsodie
hongroise \Number{12} en \kC \Sharp mineur/\kD \Flat majeur, S~244
\Number{12}.

\subsection{Valses}

Méphisto-valses~: \Number{1}, S~514~; \Number{2}, S~515~; \Number{3},
S~216.

Valses oubliées, S~215 \Number{1} et \Number{2}.

\subsection{Autres œuvres}

Ballade \Number{2} en \kB mineur, S~171.

Concerto pathétique en \kE mineur pour deux pianos, S~258.

Consolations (Six Pensées poétiques), S~172~: \Number{3} en \kD \Flat majeur
(\emph{Lento placido})~; \Number{5} en \kE majeur (\emph{Andantino}).

Fantaisie et fugue sur le thème~BACH, S~529.

\emph{In festo transfigurationis Domini nostri Jesu Christi}, S~188.

\emph{Klavierstück} en \kF \Sharp majeur, S~193.

\emph{Liebesträume} en \kA \Flat majeur, S~541 \Number{3}.

Méphisto-polka, S~217.

Nuages gris (\emph{Trübe Wolken}), S~199.

Scherzo et marche, S~177.

Sonate en \kB mineur, S~178.

\section[%
Richard Wagner (1813-1883) -- Louis Brassin]{%
\RWagner{} (1813-1883) -- \LBrassin{}}

\emph{Feuerzauber}, extrait de \emph{Die Walküre}.

\section[%
Richard Wagner (1813-1883) -- Ferenc Liszt]{%
\RWagner{} (1813-1883) -- \FLiszt{}}

\emph{Isoldes Liebestod}, extrait de \emph{Tristan und Isolde}, S~447.

\section[%
Anton Rubinštejn (1829-1894)]{%
\ARubinstein{} (1829-1894)}

Concerto pour piano et orchestre \Number{4} en \kD mineur, \Opus{70}.

\emph{Toccata} en \kD mineur, \Opus{69} \Number{5}.

\section[%
Aleksandr Borodin (1833-1887)]{%
\ABorodine{} (1833-1887)}

Petite Suite (Au couvent, en \kC \Sharp mineur~; \emph{Intermezzo}, en \kF
majeur~; Mazurka~I en \kC majeur~; Mazurka~II en \kD \Flat majeur~; Rêverie,
en \kD \Flat majeur~; Sérénade, en \kD \Flat majeur~; Nocturne, en \kG \Flat
majeur), 1878-1885.

\section[%
Johannes Brahms (1833-1897)]{%
\JBrahms{} (1833-1897)}

Ballade en \kG mineur, \Opus{118} \Number{3}.

\emph{Capriccio} en \kB mineur, \Opus{76} \Number{2}.

\emph{Intermezzi}~: en \kC \Sharp mineur, \Opus{117} \Number{3}~; en \kA
mineur, \Opus{118} \Number{1}~; en \kA majeur, \Opus{118} \Number{2}~; en
\kE \Flat mineur, \Opus{118} \Number{6}.

Deux Rhapsodies, \Opus{79}~: \Number{1} en \kB mineur~; \Number{2} en \kG
mineur.

Rhapsodie en \kE \Flat majeur, \Opus{119} \Number{4}.

\section[%
Camille Saint-Saëns (1835-1921)]{%
\CSaintSaens{} (1835-1921)}

Scherzo pour deux pianos, \Opus{87}.

\section[%
Camille Saint-Saëns (1835-1921) -- Ferenc Liszt]{%
\CSaintSaens{} (1835-1921) -- \FLiszt{}}

Danse macabre (\Opus{40}), S~555.

\section[%
Milij Balakirev (1837-1910)]{%
\MBalakirev{} (1837-1910)}

\emph{Islamey}~: Une Fantaisie orientale, \Opus{18}.

\section[%
Modest Musorgskij (1839-1881) -- Sergej Rachmaninov]{%
\MMoussorgski{} (1839-1881) -- \SRachmaninov{}}

\emph{Gopak} (ou \emph{Hopak}), extrait de l'opéra La Foire de Soročinskij.

\section[%
Pëtr Čajkovskij (1840-1893)]{%
\PTchaikovski{} (1840-1893)}

Six pièces extraites du cycle Les Saisons, \Opus{37a}~: \Number{1} Janvier
(\Quote{Au coin du feu})~; \Number{4} Avril (\Quote{Perce-neige})~;
\Number{5} Mai (\Quote{Nuits de mai})~; \Number{6} Juin
(\Quote{Barcarolle})~; \Number{10} Octobre (\Quote{Chant d'automne})~;
\Number{11} Novembre (\Quote{Trojka}).

Ruines d'un château, en \kE mineur, extrait de la suite Souvenir de Hapsal,
\Opus{2} \Number{1}.

\section[%
Pëtr Čajkovskij (1840-1893) -- Pavel Pabst]{%
\PTchaikovski{} (1840-1893) -- \PPabst{}}

Fantaisie sur des thèmes du ballet La Belle au bois dormant (\Opus{66}).

\section[%
Edvard Grieg (1843-1907)]{%
\EGrieg{} (1843-1907)}

Ballade en \kG mineur, \Opus{24}.

Nocturne en \kC majeur, extrait du livre~V des Pièces lyriques, \Opus{54}
\Number{4}.

\section[%
Anatolij Ljadov (1855-1914)]{%
\ALiadov{} (1855-1914)}

\subsection{Préludes}

Douze Préludes~: en \kB mineur, \Opus{11} \Number{1} (Morceau)~; en \kB
\Flat mineur, \Opus{31} \Number{2} (Morceau)~; en \kF \Sharp majeur,
\Opus{36} \Number{1}~; en \kG majeur, \Opus{36} \Number{3}~; en \kC mineur,
\Opus{39} \Number{2}~; en \kD mineur, \Opus{40} \Number{3}~; en \kD \Flat
majeur, \Opus{40} \Number{4}~; en \kB \Flat majeur, \Opus{46} \Number{1}~;
en \kD \Flat majeur, \Opus{57} \Number{1} (Morceau)~; autres Préludes.

\subsection{Autres œuvres}

Barcarolle en \kF \Sharp majeur, \Opus{44}.

\emph{Birjul'ki} (recueil de quatorze morceaux pour piano), \Opus{2}.

Mazurka en \kF majeur, \Opus{38}.

Mazurka en \kF mineur, \Opus{57} \Number{3} (Morceau).

Quatre Morceaux, \Opus{64}~: \Number{1} Grimace~; \Number{2} Ténèbres~;
\Number{3} Tentation~; \Number{4} Réminiscence.

\emph{Novelette} en \kC majeur, \Opus{20}.

\emph{A Musical Snuffbox} (Une Tabatière à musique) en \kA majeur,
\Opus{32}.

Valse en \kE majeur, \Opus{57} \Number{2} (Morceau).

\section[%
Sergej Taneev (1856-1915)]{%
\STaneiev{} (1856-1915)}

Prélude et fugue pour deux pianos en \kG \Sharp mineur, \Opus{29}.

\section[%
Sergej Ljapunov (1859-1924)]{%
\SLiapounov{} (1859-1924)}

\citet[p.~413]{Scriabine} mentionnent que \VSofronitsky{} a donné, durant
la saison~1938-1939, un cycle de concerts consacré à la musique russe et
soviétique~; le programme comportait, entre autres, une ou plusieurs œuvres
de \SLiapounov{}.

\section[%
Claude Debussy (1862-1918)]{%
\CDebussy{} (1862-1918)}

\subsection{\emph{Children's Corner}}

\emph{Children's Corner}, L~113 (extraits)~: \Number{I} \emph{Doctor Gradus
ad Parnassum}~; \Number{II} \emph{Jimbo's Lullaby}~; \Number{III}
\emph{Serenade of the doll}.

\subsection{Préludes}

Seize Préludes~: Danseuses de Delphes, L~117 \Number{I}~; Le Vent dans la
plaine, L~117 \Number{III}~; Les Collines d'\hbox{Anacapri}, L~117
\Number{V}~; La Fille aux cheveux de lin, L~117 \Number{VIII}~; La Danse de
Puck, L~117 \Number{XI}~; \emph{Minstrels}, L~117 \Number{XII}~; Feuilles
mortes, L~123 \Number{II}~; \emph{General Lavine -- eccentric}, L~123
\Number{VI}~; Canope, L~123 \Number{X}~; Feux d'artifice, L~123
\Number{XII}~; autres Préludes.

\subsection{Autres œuvres}

Arabesque \Number{1} en \kE majeur, L~66 \Number{I}.

En blanc et noir, pour deux pianos, L~134.

Clair de lune, L~75 \Number{III} (extrait de la Suite bergamasque).

L'\hbox{Isle} joyeuse, L~106.

Prélude en \kA mineur, L~95 \Number{I} (extrait de la suite Pour le piano).

Reflets dans l'eau, L~110 \Number{I} (extraits des Images~I).

La Soirée dans Grenade, L~100 \Number{II} (extraite des Estampes).

\section[%
Feliks Blumenfel'd (1863-1931)]{%
\FBlumenfeld{} (1863-1931)}

Deux Fragments lyriques, \Opus{47}.

\section[%
Aleksandr Glazunov (1865-1936)]{%
\AGlazounov{} (1865-1936)}

\subsection{Sonates}

Sonate \Number{1} en \kB \Flat mineur, \Opus{74}.
Sonate \Number{2} en \kE mineur, \Opus{75}.

\subsection{Autres œuvres}

Concerto pour piano et orchestre \Number{1} en \kF mineur, \Opus{92}.

Études, \Opus{31} (extraits)~: en \kC majeur, \Opus{31} \Number{1}~; en \kE
majeur \Quote{La Nuit}, \Opus{31} \Number{3}.

Gavotte en \kD majeur, \Opus{49} \Number{3} (Morceau).

Prélude en \kD \Flat majeur, \Opus{49} \Number{1} (Morceau).

Préludes et fugues~: en \kD mineur, \Opus{62}~; en \kA mineur, \Opus{101}
\Number{1}~; en \kC majeur, \Opus{101} \Number{4}.

\section[%
Ferruccio Busoni (1866-1924)]{%
\FBusoni{} (1866-1924)}

\emph{Duettino concertante}, BV~B~88, d'après le finale du Concerto pour
piano et orchestre \Number{19} en \kF majeur, K~459, de \Mozart{}.

\section[%
Aleksandr Skrjabin (1872-1915)]{%
\AScriabine{} (1872-1915)}

\subsection{Danses}

Danse languide, \Opus{51} \Number{4} (Morceau).

Deux Danses, \Opus{73}~: \Number{1} Guirlandes~; \Number{2} Flammes sombres.

\subsection{Études}

\Comment{Les vingt-six Études d'\AScriabine{} figuraient au répertoire de
concert de \VSofronitsky{}.}

Étude en \kC \Sharp mineur, \Opus{2} \Number{1} (Morceau).

Douze Études, \Opus{8}~: \Number{1} en \kC \Sharp majeur~; \Number{2} en \kF
\Sharp mineur~; \Number{3} en \kB mineur~; \Number{4} en \kB majeur~;
\Number{5} en \kE majeur~; \Number{6} en \kA majeur~; \Number{7} en \kB
\Flat mineur~; \Number{8} en \kA \Flat majeur~; \Number{9} en \kG \Sharp
mineur~; \Number{10} en \kD \Flat majeur~; \Number{11} en \kB \Flat mineur~;
\Number{12} en \kD \Sharp mineur.

Huit Études, \Opus{42}~: \Number{1} en \kD \Flat majeur~; \Number{2} en \kF
\Sharp mineur~; \Number{3} en \kF \Sharp majeur~; \Number{4} en \kF \Sharp
majeur~; \Number{5} en \kC \Sharp mineur~; \Number{6} en \kD \Flat majeur~;
\Number{7} en \kF mineur~; \Number{8} en \kE \Flat majeur.

Étude en \kE \Flat majeur, \Opus{49} \Number{1} (Morceau).

Étude, \Opus{56} \Number{4} (Morceau).

Trois Études, \Opus{65}.

\subsection{Impromptus}

Impromptu à la Mazur en \kC majeur, \Opus{2} \Number{3} (Morceau).

Impromptu en \kB \Flat mineur, \Opus{12} \Number{2}.

Impromptu en \kF \Sharp mineur, \Opus{14} \Number{2}.

\subsection{Mazurkas}

Impromptu à la Mazur en \kC majeur, \Opus{2} \Number{3} (Morceau).

Dix Mazurkas, \Opus{3}~: \Number{1} en \kB mineur~; \Number{2} en \kF \Sharp
mineur~; \Number{3} en \kG mineur~; \Number{4} en \kE majeur~; \Number{5} en
\kD \Sharp mineur~; \Number{6} en \kC \Sharp mineur~; \Number{7} en \kE
mineur~; \Number{8} en \kB \Flat mineur~; \Number{9} en \kG \Sharp mineur~;
\Number{10} en \kE \Flat mineur.

Mazurkas, \Opus{25} (extraits)~: \Number{3} en \kE mineur~; \Number{7} en
\kF \Sharp mineur~; \Number{8} en \kB majeur.

Deux Mazurkas, \Opus{40}~: \Number{1} en \kD \Flat majeur~; \Number{2} en
\kF \Sharp majeur.

\subsection{Morceaux}

\Comment{Les vingt-quatre Morceaux d'\AScriabine{} figuraient au répertoire
de concert de \VSofronitsky{}~; les données de ce paragraphe répètent celles
d'autres paragraphes.}

Trois Morceaux, \Opus{2}~: \Number{1} Étude en \kC \Sharp mineur~;
\Number{2} Prélude en \kB majeur~; \Number{3} Impromptu à la Mazur en \kC
majeur.

Trois Morceaux, \Opus{45}~: \Number{1} Feuillet d'album en \kE \Flat
majeur~; \Number{2} Poème fantastique en \kC majeur~; \Number{3} Prélude en
\kE \Flat majeur.

Trois Morceaux, \Opus{49}~: \Number{1} Étude en \kE \Flat majeur~;
\Number{2} Prélude en \kF majeur~; \Number{3} Rêverie en \kC majeur.

Quatre Morceaux, \Opus{51}~: \Number{1} Fragilité~; \Number{2} Prélude en
\kA mineur~; \Number{3} Poème ailé~; \Number{4} Danse languide.

Trois Morceaux, \Opus{52}~: \Number{1} Poème en \kC majeur~; \Number{2}
Énigme~; \Number{3} Poème languide en \kB majeur.

Quatre Morceaux, \Opus{56}~: \Number{1} Prélude en \kE \Flat majeur~;
\Number{2} Ironies en \kC majeur~; \Number{3} Nuances~; \Number{4} Étude.

Deux Morceaux, \Opus{57}~: \Number{1} Désir~; \Number{2} Caresse dansée.

Deux Morceaux, \Opus{59}~: \Number{1} Poème~; \Number{2} Prélude.

\subsection{Nocturnes}

Deux Nocturnes, \Opus{5}~: \Number{1} en \kF \Sharp mineur~; \Number{2} en
\kA majeur.

Nocturne pour la main gauche en \kD \Flat majeur, \Opus{9} \Number{2}.

Poème-nocturne, \Opus{61}.

\subsection{Poèmes}

\Comment{Les vingt Poèmes d'\AScriabine{} figuraient au répertoire de
concert de \VSofronitsky{}.}

Deux Poèmes, \Opus{32}~: \Number{1} en \kF \Sharp majeur~; \Number{2} en \kD
majeur.

Poème tragique, \Opus{34}.

Poème satanique, \Opus{36}.

Poème, \Opus{41}.

Deux Poèmes, \Opus{44}~: \Number{1} en \kC majeur~; \Number{2} en \kC
majeur.

Poème fantastique en \kC majeur, \Opus{45} \Number{2} (Morceau).

Poème ailé, \Opus{51} \Number{3} (Morceau).

Poème en \kC majeur, \Opus{52} \Number{1} (Morceau).

Poème languide en \kB majeur, \Opus{52} \Number{3} (Morceau).

Poème, \Opus{59} \Number{1} (Morceau).

Poème-nocturne, \Opus{61}.

Deux Poèmes, \Opus{63}~: \Number{1} Masque~; \Number{2} Étrangeté.

Deux Poèmes, \Opus{69}~: \Number{1} \emph{Allegretto}~; \Number{2}
\emph{Allegretto}.

Deux Poèmes, \Opus{71}.

Vers la flamme, \Opus{72} (Poème).

\subsection{Préludes}

\Comment{Les nonante Préludes d'\AScriabine{} figuraient au répertoire de
concert de \VSofronitsky{}.}

Prélude en \kB majeur, \Opus{2} \Number{2} (Morceau).

Prélude pour la main gauche en \kC \Sharp mineur, \Opus{9} \Number{1}.

Vingt-quatre Préludes, \Opus{11}~: \Number{1} en \kC majeur~; \Number{2} en
\kA mineur~; \Number{3} en \kG majeur~; \Number{4} en \kE mineur~;
\Number{5} en \kD majeur~; \Number{6} en \kB mineur~; \Number{7} en \kA
majeur~; \Number{8} en \kF \Sharp mineur~; \Number{9} en \kE majeur~;
\Number{10} en \kC \Sharp mineur~; \Number{11} en \kB majeur~; \Number{12}
en \kG \Sharp mineur~; \Number{13} en \kG \Flat majeur~; \Number{14} en \kE
\Flat mineur~; \Number{15} en \kD \Flat majeur~; \Number{16} en \kB \Flat
mineur~; \Number{17} en \kA \Flat majeur~; \Number{18} en \kF mineur~;
\Number{19} en \kE \Flat majeur~; \Number{20} en \kC mineur~; \Number{21} en
\kB \Flat majeur~; \Number{22} en \kG mineur~; \Number{23} en \kF majeur~;
\Number{24} en \kD mineur.

Six Préludes, \Opus{13}~: \Number{1} en \kC majeur~; \Number{2} en \kA
mineur~; \Number{3} en \kG majeur~; \Number{4} en \kE mineur~; \Number{5} en
\kD majeur~; \Number{6} en \kB mineur.

Cinq Préludes, \Opus{15}~: \Number{1} en \kA majeur~; \Number{2} en \kF
\Sharp mineur~; \Number{3} en \kE majeur~; \Number{4} en \kE majeur~;
\Number{5} en \kC \Sharp mineur.

Cinq Préludes, \Opus{16}~: \Number{1} en \kB majeur~; \Number{2} en \kG
\Sharp mineur~; \Number{3} en \kG \Flat majeur~; \Number{4} en \kE \Flat
mineur~; \Number{5} en \kF \Sharp majeur.

Sept Préludes, \Opus{17}~: \Number{1} en \kD mineur~; \Number{2} en \kE
\Flat majeur~; \Number{3} en \kD \Flat majeur~; \Number{4} en \kB \Flat
mineur~; \Number{5} en \kF mineur~; \Number{6} en \kB \Flat majeur~;
\Number{7} en \kG mineur.

Quatre Préludes, \Opus{22}~: \Number{1} en \kG \Sharp mineur~; \Number{2} en
\kC \Sharp mineur~; \Number{3} en \kB majeur~; \Number{4} en \kB mineur.

Deux Préludes, \Opus{27}~: \Number{1} en \kG mineur~; \Number{2} en \kB
majeur.

Quatre Préludes, \Opus{31}~: \Number{1} en \kD \Flat majeur~; \Number{2} en
\kF \Sharp mineur~; \Number{3} en \kE \Flat mineur~; \Number{4} en \kC
majeur.

Quatre Préludes, \Opus{33}~: \Number{1} en \kE majeur~; \Number{2} en \kF
\Sharp majeur~; \Number{3} en \kC majeur~; \Number{4} en \kA \Flat majeur.

Trois Préludes, \Opus{35}~: \Number{1} en \kD \Flat majeur~; \Number{2} en
\kB \Flat majeur~; \Number{3} en \kC majeur.

Quatre Préludes, \Opus{37}~: \Number{1} en \kB \Flat mineur~; \Number{2} en
\kF \Sharp majeur~; \Number{3} en \kB majeur~; \Number{4} en \kG mineur.

Quatre Préludes, \Opus{39}~: \Number{1} en \kF \Sharp majeur~; \Number{2} en
\kD majeur~; \Number{3} en \kG majeur~; \Number{4} en \kA \Flat majeur.

Prélude en \kE \Flat majeur, \Opus{45} \Number{3} (Morceau).

Quatre Préludes, \Opus{48}~: \Number{1} en \kF \Sharp majeur~; \Number{2} en
\kC majeur~; \Number{3} en \kD \Flat majeur~; \Number{4} en \kC majeur.

Prélude en \kF majeur, \Opus{49} \Number{2} (Morceau).

Prélude en \kA mineur, \Opus{51} \Number{2} (Morceau).

Prélude en \kE \Flat majeur, \Opus{56} \Number{1} (Morceau).

Prélude, \Opus{59} \Number{2} (Morceau).

Deux Préludes, \Opus{67}~: \Number{1} \emph{Andante}~; \Number{2}
\emph{Presto}.

Cinq Préludes, \Opus{74}~: \Number{1} Douloureux, déchirant~; \Number{2}
Très lent, contemplatif~; \Number{3} \emph{Allegro drammatico}~; \Number{4}
Lent, vague, indécis~; \Number{5} Fier, belliqueux.

\subsection{Sonates}

\Comment{\VSofronitsky{} n'a joué que le quatrième mouvement de la Première
Sonate, \Opus{6}, lors du concert du~16 mai~1957 au musée \Scriabine{}, mais
l'œuvre entière semble avoir été jouée le~6 novembre~1926 lors d'un concert
à la petite salle du conservatoire de Moskva.
En revanche, le pianiste s'est toujours refusé à jouer, du moins en public,
la Septième Sonate, \Opus{64} (\Quote{Messe blanche}).}

Sonates~: \Number{1} en \kF mineur, \Opus{6}~; \Number{2} en \kG \Sharp
mineur, \Opus{19} (Sonate-fantaisie)~; \Number{3} en \kF \Sharp mineur,
\Opus{23} (\Quote{États d'âme})~; \Number{4} en \kF \Sharp majeur,
\Opus{30}~; \Number{5}, \Opus{53}~; \Number{6}, \Opus{62}~; \Number{8},
\Opus{66}~; \Number{9}, \Opus{68} (\Quote{Messe noire})~; \Number{10},
\Opus{70}.

\subsection{Valses}

Valse en \kA \Flat majeur, \Opus{38}.

Quasi-valse en \kF majeur, \Opus{47}.

\subsection{Autres œuvres}

\Comment{Le classement des autres œuvres d'\AScriabine{} est croissant par
ordre de numéros d'opus.}

\emph{Allegro appassionato} en \kE \Flat mineur, \Opus{4}.

\emph{Allegro} de concert en \kB \Flat mineur, \Opus{18}.

Concerto pour piano en \kF \Sharp mineur, \Opus{20}.

Polonaise en \kB \Flat mineur, \Opus{21}.

Fantaisie en \kB mineur, \Opus{28}.

Feuillet d'album en \kE \Flat majeur, \Opus{45} \Number{1} (Morceau).

Scherzo, \Opus{46}.

Rêverie en \kC majeur, \Opus{49} \Number{3} (Morceau).

Fragilité, \Opus{51} \Number{1} (Morceau).

Énigme, \Opus{52} \Number{2} (Morceau).

Ironies en \kC majeur, \Opus{56} \Number{2} (Morceau).

Nuances, \Opus{56} \Number{3} (Morceau).

Désir, \Opus{57} \Number{1} (Morceau).

Caresse dansée, \Opus{57} \Number{2} (Morceau).

Feuillet d'album, \Opus{58}.

Prométhée~: Le Poème du feu, \Opus{60} (partie de piano).

\section[%
Sergej Rachmaninov (1873-1943)]{%
\SRachmaninov{} (1873-1943)}

\subsection{Études-tableaux}

Six Études-tableaux extraites de l'\Opus{33}~ \Number{2} en \kC majeur~;
\Number{6} en \kE \Flat majeur~; \Number{7} en \kG mineur~; \Number{8} en
\kC \Sharp mineur~; autres œuvres.

Cinq Études-tableaux extraites de l'\Opus{39}~: \Number{2} en \kA mineur~;
\Number{4} en \kB mineur~; \Number{5} en \kE \Flat mineur~; \Number{6} en
\kA mineur~; \Number{9} en \kD majeur.

\subsection{Moments musicaux}

Cinq Moments musicaux extraits de l'\Opus{16}~: \Number{2} en \kE \Flat
mineur (\emph{Allegretto})~; \Number{3} en \kB mineur (\emph{Andante
cantabile})~; \Number{4} en \kE mineur (\emph{Presto})~; \Number{5} en \kD
\Flat majeur (\emph{Adagio sostenuto})~; \Number{6} en \kC majeur
(\emph{Maestoso}).

\subsection{Préludes}

Prélude en \kC \Sharp mineur, \Opus{3} \Number{2} (Morceau de fantaisie).

Huit Préludes extraits de l'\Opus{23}~: \Number{1} en \kF \Sharp mineur~;
\Number{2} en \kB \Flat majeur~; \Number{4} en \kD majeur~; \Number{5} en
\kG mineur~; \Number{6} en \kE \Flat majeur~; \Number{7} en \kC mineur~;
\Number{8} en \kA \Flat majeur~; \Number{9} en \kE \Flat mineur.

Six Préludes extraits de l'\Opus{32}~: \Number{1} en \kC majeur~; \Number{2}
en \kB \Flat mineur~; \Number{3} en \kE majeur~; \Number{5} en \kG majeur~;
\Number{8} en \kA mineur~; \Number{12} en \kG \Sharp mineur.

\subsection{Autres œuvres}

Concerto pour piano \Number{1} en \kF \Sharp mineur, \Opus{1}.
À l'époque des années d'études de \VSofronitsky{}, ce Concerto était joué
dans la version originale~: voir \citet[p.~81, note~10]{White}.

\emph{Humoresque} en \kG majeur, \Opus{10} \Number{5} (Morceau de salon).

\emph{Oriental Sketch} en \kB \Flat majeur (Morceau, 1917).

Polichinelle, \Opus{3} \Number{4} (Morceau de fantaisie).

Polka de~W.R. (initiales de la transcription en allemand du nom du père du
compositeur, Wassily Rachmaninoff).
Il s'agit en fait de la transcription de l'\Opus{303} de Franz Behr.

Sérénade, \Opus{3} \Number{5} (Morceau de fantaisie).

Valse -- peut-être l'\Opus{10} \Number{2} en \kA majeur (Morceau de salon).

\section[%
Maurice Ravel (1875-1937)]{%
\MRavel{} (1875-1937)}

Sonatine.

\section[%
Leonid Nikolaev (1878-1942)]{%
\LNikolaiev{} (1878-1942)}

Variations pour deux pianos sur un thème de quatre notes.

\section[%
Nikolaj Metner (1880-1951)]{%
\NMedtner{} (1880-1951)}

\subsection{Contes de fées (\emph{Skazki})}

Deux Contes de fées, \Opus{8}~: \Number{1} en \kC mineur~; \Number{2} en \kC
mineur.

Deux Contes de fées extraits de l'\Opus{9}.

Un Conte de fées extrait de l'\Opus{14}~: \Number{2} en \kE mineur (La
Marche du paladin).

Deux Contes de fées, \Opus{20}~: \Number{1} en \kB \Flat mineur~; \Number{2}
en \kB mineur (\emph{Campanella}).

Quatre Contes de fées, \Opus{26}~: \Number{1} en \kE \Flat majeur~;
\Number{2} en \kE \Flat majeur~; \Number{3} en \kF mineur~; \Number{4} en
\kF \Sharp mineur.

Quatre Contes de fées, \Opus{34}~: \Number{1} en \kB mineur (Le Violon
magique)~; \Number{2} en \kE mineur~; \Number{3} en \kA mineur (Lutin de
bois)~; \Number{4} en \kD mineur.

\subsection{Sonates}

Sonate en \kA \Flat majeur, \Opus{11} \Number{1}.
Sonate en \kD mineur, \Opus{11} \Number{2} (Sonate Élégie).
Sonate en \kC majeur, \Opus{11} \Number{3}.
Sonate en \kG mineur, \Opus{22}.

\subsection{Autres œuvres}

Une Arabesque extraite de l'\Opus{7}~: \Number{1} en \kB mineur (Idylle).

Une Dithyrambe extraite de l'\Opus{10}.

Une Improvisation fantastique extraite de l'\Opus{2}~: \Number{3}
\emph{Scherzo infernale}.

Quatre Morceaux pour piano, \Opus{4}~: \Number{1} Étude en \kG \Sharp
mineur~; \Number{2} Caprice en \kC majeur~; \Number{3} Moment musical en \kC
mineur~; \Number{4} Prélude en \kE \Flat majeur.

Un Morceau extrait de l'\Opus{31}~: \Number{2} en \kB mineur (Marche
funèbre).

Une \emph{Novelette} extraite de l'\Opus{17}~: \Number{2} en \kC mineur.

\section[%
Nikolaj Mjaskovskij (1881-1950)]{%
\NMiaskovski{} (1881-1950)}

Fantaisies (\emph{Pričudy}), six esquisses pour piano, \Opus{25}.

Sonate pour piano \Number{2} en \kF \Sharp mineur, en un mouvement,
\Opus{13}.

\section[%
Nikolaj Mjaskovskij (1881-1950) -- Anna Aljavdina]{%
\NMiaskovski{} (1881-1950) -- \AAlyavdina{}}

Scherzo de la Symphonie \Number{5} en \kD majeur, \Opus{18}.

\section[%
Anatolij Aleksandrov (1888-1982)]{%
\AAleksandrov{} (1888-1982)}

\citet[p.~413]{Scriabine} mentionnent que \VSofronitsky{} a donné, durant
la saison~1938-1939, un cycle de concerts consacré à la musique russe et
soviétique~; le programme comportait, entre autres, une ou plusieurs œuvres
d'\AAleksandrov{}.

\section[%
Vladimir Ščerbačëv (1889-1952)]{%
\VChtcherbatchiov{} (1889-1952)}

\citet[p.~413]{Scriabine} mentionnent que \VSofronitsky{} a donné, durant
la saison~1938-1939, un cycle de concerts consacré à la musique russe et
soviétique~; le programme comportait, entre autres, une ou plusieurs œuvres
de \VChtcherbatchiov{}.

\section[%
Samuil Fejnberg (1890-1962)]{%
\SFeinberg{} (1890-1962)}

\citet[p.~413]{Scriabine} mentionnent que \VSofronitsky{} a donné, durant
la saison~1938-1939, un cycle de concerts consacré à la musique russe et
soviétique~; le programme comportait, entre autres, une ou plusieurs œuvres
de \SFeinberg{}.

\section[%
Sergej Prokof'ev (1891-1953)]{%
\SProkofiev{} (1891-1953)}

\subsection{Contes de la vieille grand-mère}

Quatre Contes de la vieille grand-mère, \Opus{31}~: \Number{1}
\emph{Moderato} en \kD mineur~; \Number{2} \emph{Andantino} en \kF \Sharp
mineur~; \Number{3} \emph{Andante assai} en \kE mineur~; \Number{4}
\emph{Sostenuto} -- \emph{Pocchissimo più animato} -- \emph{Molto andante}
en \kB mineur.

\subsection{Pièces, \Opus{12 et~32}}

Dix Pièces pour piano, \Opus{12}~: \Number{1} Marche en \kF \Sharp mineur~;
\Number{2} Gavotte en \kG mineur~; \Number{3} Rigaudon en \kC majeur~;
\Number{4} Mazurka en \kB majeur~; \Number{5} \emph{Capriccio} en \kE
mineur~; \Number{6} Légende en \kD mineur~; \Number{7} Prélude en \kC
majeur~; \Number{8} Allemande en \kF \Sharp mineur~; \Number{9} Scherzo
humoristique en \kC majeur~; \Number{10} Scherzo en \kA mineur.

Quatre Pièces pour piano, \Opus{32}~: \Number{1} Danse~; \Number{2} Menuet~;
\Number{3} Gavotte~; \Number{4} Valse.

\subsection{Sarcasmes}

Cinq Sarcasmes, \Opus{17}~: \Number{1} \emph{Tempestoso}~; \Number{2}
\emph{Allegro rubato} -- \emph{Più mosso}~; \Number{3} \emph{Allegro
precipitato} -- \emph{Un poco largamente}~; \Number{4} \emph{Smanioso}~;
\Number{5} \emph{Precipitosissimo}.

\subsection{Sonates}

Sonate pour piano \Number{2} en \kD mineur, \Opus{14}.
Sonate pour piano \Number{3} en \kA mineur, \Opus{28} (\Quote{D'après de
vieux cahiers}).
Sonate pour piano \Number{7} en \kB \Flat majeur, \Opus{83}
(\Quote{Stalingrad}).

\subsection{Visions fugitives}

Au moins quinze pièces extraites des Visions fugitives, \Opus{22}~:
\Number{1} \emph{Lentamente}~; \Number{2} \emph{Andante}~; \Number{3}
\emph{Allegretto}~; \Number{4} \emph{Animato}~; \Number{5} \emph{Molto
giocoso}~; \Number{6} \emph{Con eleganza}~; \Number{7} \emph{Pittoresco}
(\Quote{\emph{Arpa}})~; \Number{10} \emph{Ridicolosamente}~; \Number{11}
\emph{Con vivacità}~; \Number{12} \emph{Assai moderato}~; \Number{17}
\emph{Poetico}~; \Number{18} \emph{Con una dolce lentezza}~; autres pièces.

\subsection{Autres œuvres}

Deux Études pour piano extraites de l'\Opus{2}.

Trois Pièces pour piano extraites de l'\Opus{3}~: \Number{1} Conte~;
\Number{3} Marche~; \Number{4} Fantômes.

Deux Pièces pour piano extraites de l'\Opus{4}~: \Number{1} Réminiscence~;
\Number{4} Suggestion diabolique.

\emph{Toccata} pour piano en \kD mineur, \Opus{11}.

Marche et Scherzo pour piano d'après l'opéra L'\hbox{Amour} des trois
oranges, \Opus{33ter} (transcription par \Prokofiev{})~: \Number{1} Marche~;
\Number{2} Scherzo.

Ouverture sur des thèmes juifs en \kC mineur pour clarinette, quatuor à
cordes et piano, \Opus{34}.

Pensées (Trois Pièces pour piano), \Opus{62}~: \Number{1} \emph{Adagio
penseroso}~; \Number{2} \emph{Lento}~; \Number{3} \emph{Andante}.

Dix Pièces pour piano sur le ballet \emph{Roméo et Juliette}, \Opus{75}~:
\emph{Montaigu et Capulet}~; \emph{Scène}~; \emph{Menuet}~;
\emph{Mercutio}~; \emph{Frère Laurent}~; \emph{Juliette et sa nourrice}~;
autres pièces.

\section[%
Francis Poulenc (1899-1963)]{%
\FPoulenc{} (1899-1963)}

Trois Promenades extraites du recueil~FP.024~: \Number{II} En auto~;
\Number{IV} En bateau~; \Number{VIII} En chemin de fer.

\section[%
Vladimir Krjukov (1902-1960)]{%
\VKrioukov{} (1902-1960)}

Rhapsodie russe.

\section[%
Valerian Bogdanov-Berezovskij (1903-1971)]{%
\VBogdanovBerezovsky{} (1903-1971)}

Deux Études, \Opus{16}.

Préludes.

\emph{Grave} extrait de la Sonate pour piano.

Petite Suite~: \Number{1} Danse en \kA majeur~; \Number{2} Marche en \kD
majeur.

\section[%
Dmitrij Kabalevskij (1904-1987)]{%
\DKabalevski{} (1904-1987)}

Sonatine pour piano \Number{1} en \kC majeur, \Opus{13} \Number{1}.

Douze Préludes.

\section[%
Dmitrij Šostakovič (1906-1975)]{%
\DChostakovitch{} (1906-1975)}

Au moins dix Préludes extraits de l'\Opus{34}~: \Number{10} en \kC \Sharp
mineur (\emph{Moderato non troppo})~; \Number{13} en \kF \Sharp majeur
(\emph{Moderato})~; autres préludes.

Cinq Préludes et fugues extraits de l'\Opus{87}~: \Number{1} en \kC majeur~;
\Number{2} en \kA mineur~; \Number{3} en \kG majeur~; \Number{7} en \kA
majeur~; \Number{9} en \kE majeur.

\section[%
Boris Gol'c (1913-1942)]{%
\BGoltz{} (1913-1942)}

Six Préludes extraits de l'\Opus{2}, y compris le Prélude \Number{4} en \kE
mineur (\emph{Andante con anima}).

Scherzo en \kE mineur.
