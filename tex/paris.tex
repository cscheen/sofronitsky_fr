\chapter[%
Vladimir Sofronickij à Paris (1928-05-31 -- 1929-12-20)][%
Vladimir Sofronickij à Paris]{%
\VSofronitsky{} à Paris (1928-05-31 -- 1929-12-20)}
\label{chap:Paris}

Lors de son unique tournée importante à l'étranger, après un récital donné à
Warszawa le~20 mars~1928 \citep[voir][p.~44]{White}, \VSofronitsky{} arriva
à Paris vers la fin du mois d'avril~1928 \citep[voir][p.~112]{White}~; il
rentra cependant en Rossija le~27 janvier~1930 \citep[voir][p.~45]{White}.
Pendant ce séjour parisien, entre le~31 mai~1928 et le~20 décembre~1929,
\VSofronitsky{} donna une série de sept récitals précédés d'un récital privé
pour des hôtes invités \citep[voir][p.~44-45, pour les programmes]{White}.
Quelques revues musicales ont publié leurs opinions sur le jeune pianiste
russe \citep[voir][p.~83]{Semaine334a}.
Dans les citations suivantes, la translittération scientifique ne s'applique
pas~; les textes reproduits respectent la transcription adoptée dans les
articles originaux.

\begin{quotation}
 \Quote{Je connais bien Vladimir Sofronitzky.
 C'est un des plus remarquables parmi les jeunes pianistes russes.
 Son jeu se distingue par la maturité artistique, la perfection technique,
 la profondeur, expression et éclat de grande envergure.
 Je ne crains pas d'affirmer qu'un grand avenir artistique s'ouvre devant
 lui.}%
 \sourceatright{\citep{Glazounov}}
\end{quotation}

\begin{quotation}
 \emph{Comœdia}, 26~juin~1928~:
 \Quote{...~Nous nous trouvons en présence d'un artiste dont les
 interprétations sont d'une remarquable musicalité.
 Son jeu très coloré sait allier la puissance et la finesse et son style
 est toujours du meilleur aloi.
 Il a remporté un succès très vif et très mérité.}%
 \sourceatright{\citep{Messager}}
\end{quotation}

\begin{quotation}
 \emph{Le Figaro}, 26~juin~1928~:
 \Quote{Parmi les pianistes, il faut noter le début remarquable de
 M.~Sofronitzki, dont les deux récitals témoignèrent d'un acquis technique
 fort solide et, ce qui est plus rare, d'une nature profondément musicale.
 Chopin et Scriabine eurent en lui un traducteur vibrant et compréhensif.}%
 \sourceatright{\citep{Golestan28a}}
\end{quotation}

\begin{quotation}
 \emph{Les Dernières Nouvelles} (édition en russe), 26~juin~1928~:
 \Quote{Les traits caractéristiques du jeu de V.~\Sofronitsky{} -- la
 passion, l'élan et en même temps le lyrisme nécessaire -- sont fascinants
 et naturels à l'égard des romantiques~: \Liszt{}, \Chopin{}, \Scriabine{}.
 Pour ce dernier compositeur, je dois avouer que depuis la mort du génial
 créateur de \emph{Prométhée} personne ne pouvait me satisfaire dans
 l'exécution des œuvres de \Scriabine{}.
 Après de longues années, pour la première fois V.~\Sofronitsky{} a invoqué
 devant nous aujourd'hui \Scriabine{}.}%
 \sourceatright{%
 \citep[Boris de Schlœzer, cité par][p.~369-370]{Nikonovich08}}
\end{quotation}

\begin{quotation}
 \emph{Excelsior}, 28~juin~1928~:
 \Quote{M.~Sofronitzky épanouit généreusement les dons d'une riche nature,
 en même temps qu'il manifeste des connaissances opportunément dominées.
 Ce jeune artiste a, certes, devant lui de belles perspectives.}%
 \sourceatright{\citep{Tromp}}
\end{quotation}

\begin{quotation}
 \emph{Courrier musical}, 1\up{er}~juillet~1928~:
 \Quote{Pianiste puissant, virtuose de grande classe, M.~Vladimir
 Sofronitzky s'impose par des qualités incontestables, un acquis
 magnifique...}%
 \sourceatright{\citep{Ple}}
\end{quotation}

\begin{quotation}
 \emph{Courrier musical}, Petr Wolf, 1\up{er}~août~1928~:
 \Quote{V.~\Sofronitsky{} est un pianiste inspiré doué d'une merveilleuse
 variété de sonorité, dont il use habilement, lorsque cela est nécessaire,
 plus faible ou, au contraire, rendue plus puissante.
 Son interprétation est passionnée, fantasque.}%
 \sourceatright{\citep{Wolf28}}
\end{quotation}

\begin{quotation}
 \emph{La Semaine à Paris}, 26~octobre~1928~:
 \Quote{M.~Vladimir Sofronitzky, que l'on entendra en deux récitals à la
 salle Chopin (le~29 octobre dans du Schumann, Liszt, Medtner, Prokofieff,
 Ravel et Scriabine, le~5 novembre dans du Chopin), mérite l'attention.
 Il offre un ensemble de qualités qui font les grands pianistes, notamment
 la puissance de sonorité, la profondeur d'expression, un mécanisme
 étincelant, un sens fort exact de la couleur.
 Son apparition à Paris à la fin du printemps dernier fut justement
 remarquée.
 C'était la première fois que M.~Vladimir Sofronitzky quittait la Russie,
 son pays natal, où il jouit déjà d'une éclatante popularité.
 Un avenir bleu et or s'ouvre sous toutes les latitudes devant ce valeureux
 jeune homme qui, je le signale entre parenthèses, est le gendre de
 l'illustre compositeur du \emph{Poème de l'\hbox{Extase}}.}%
 \sourceatright{\citep{CarolBerard}}
\end{quotation}

\begin{quotation}
 \emph{Le Figaro}, 2~novembre~1928~:
 \Quote{Tandis que M.~Vladimir Sofronitzky affirme des dons rares que nous
 avons déjà signalés.
 C'est un virtuose de grand avenir, qui sait unir l'aisance d'une technique
 homogène avec des recherches personnelles qui n'excluent ni la fantaisie,
 ni la belle vision musicale.}%
 \sourceatright{\citep{Golestan28b}}
\end{quotation}

\begin{quotation}
 \emph{Gazette musicale}~:
 \Quote{Le pianiste V.~\Sofronitsky{} a donné, salle \Pleyel{}, un
 remarquable programme de récital~: \Beethoven{} (Sonate op.~111),
 \Schumann{} («~Carnaval~»), \Chopin{}, \Liszt{}, \Schubert{}.
 \Sofronitsky{} fait preuve d'une virtuosité rare et remarquable et propose
 une exécution brillante de nombreuses œuvres de la littérature pianistique.
 C'est un excellent pianiste, non seulement pour le présent, mais pour le
 futur.}%
 \sourceatright{\citep[p.~370]{Nikonovich08}}
\end{quotation}

\section{Récital privé, 1928-05-31 (jeudi)}
\label{rec:Paris0}

Ce récital a eu lieu à la salle \Debussy{} de la maison \Pleyel{}, rue Daru,
\Number{8}.
Son programme était le suivant \citep[voir][]{White}~:
\begin{itemize}
 \item
 \Beethoven{}~: Sonate \Number{32} en \kC mineur, \Opus{111}~;
 \item
 \Schumann{}~: Fantaisie en \kC majeur, \Opus{17}~;
 \item
 \Scriabine{}~: Sonate \Number{5}, \Opus{53}~;
 \item
 en bis, plusieurs œuvres de \Chopin{} et de \Scriabine{}.
\end{itemize}

\section{Premier récital, 1928-06-18 (lundi)}
\label{rec:Paris1}

Ce récital a eu lieu à~21~heures, à la salle des agriculteurs, rue
d'\hbox{Athènes}, \Number{8}.
Selon \citet[p.~400]{Scriabine}, il s'agit plutôt de la salle \Chopin{}.
Il ne semble pas que son programme soit connu en Occident
\citep[voir][]{White}, du moins de façon entière \citep[voir][p.~61]{Juban}.
Le programme complet est donné par \citet[p.~150]{Nekrasova08} et
\citet[p.~400]{Scriabine}~:
\begin{itemize}
 \item
 \Liszt{}~: Sonate en \kB mineur, S~178~;
 \item
 \Chopin{}~: Ballade en \kG mineur, \Opus{23}~; Nocturne en \kF majeur,
 \Opus{15} \Number{1}~; Trois Préludes~; Barcarolle en \kF \Sharp majeur,
 \Opus{60}~; Deux Mazurkas~; Scherzo en \kB \Flat mineur, \Opus{31}~;
 \item
 \Scriabine{}~: Étude en \kD \Sharp mineur, \Opus{8} \Number{12}~; Étude en
 \kD \Flat majeur, \Opus{8} \Number{10}~; Un Poème de l'\Opus{32}~;
 Préludes, \Opus{74}~; Sonate \Number{5}, \Opus{53}.
\end{itemize}

\section{Deuxième récital, 1928-06-25 (lundi)}
\label{rec:Paris2}

Ce récital a eu lieu à~21~heures, à la salle des agriculteurs, rue
d'\hbox{Athènes}, \Number{8}.
Selon le fils du pianiste cité par \citet[p.~400]{Scriabine}, il s'agit
plutôt de la salle \Chopin{}.
Il ne semble pas que son programme soit connu en Occident
\citep[voir][]{Heugel4808, White}, mais le programme complet est donné par
\citet[p.~150]{Nekrasova08}~:
\begin{itemize}
 \item
 \Beethoven{}~: Sonate \Number{14} en \kC \Sharp mineur, \Opus{27}
 \Number{2}~; Sonate \Number{23} en \kF mineur, \Opus{57}~;
 \item
 \Schumann{}~: Carnaval, \Opus{9}~;
 \item
 \Poulenc{}~: Trois Promenades, FP~24 \Number{2}, \Number{4} et \Number{8}~;
 \item
 \Prokofiev{}~: Trois Sarcasmes de l'\Opus{17}~;
 \item
 \Liszt{}~: Méphisto-valse.
\end{itemize}

\section{Troisième récital, 1928-10-29 (lundi)}
\label{rec:Paris3}

Ce récital a eu lieu à~21~heures, à la salle \Chopin{} de la maison
\Pleyel{}, rue Daru, \Number{8}.
Selon \citet[p.~401]{Scriabine}, il s'agit plutôt de la salle \Debussy{}~;
le fils du pianiste indique la salle \Chopin{}.
Son programme était le suivant \citep[voir][]{Nekrasova08, Semaine334b,
Semaine335, White}~:
\begin{itemize}
 \item
 \Schumann{}~: Études symphoniques, \Opus{13}~;
 \item
 \Liszt{}~: \emph{Sposalizio}, S~161 \Number{1}~; \emph{Canzonetta del
 Salvator Rosa}, S~161 \Number{3}~; \emph{Il penseroso}, S~161 \Number{2}~;
 Méphisto-valse~;
 \item
 \Medtner{}~: Trois \emph{Skazki}, \Opus{26} (\Number{3} en \kF mineur) et
 \Opus{20} (\Number{1} en \kB \Flat mineur et \Number{2} en \kB mineur)~;
 \item
 \Prokofiev{}~: Trois Visions fugitives de l'\Opus{22}~; Sonate \Number{3}
 en \kA mineur, \Opus{28}~;
 \item
 \Ravel{}~: Sonatine~;
 \item
 \Scriabine{}~: Poème et Énigme de l'\Opus{52}~; Poème satanique, \Opus{36}.
\end{itemize}

\section{Quatrième récital, 1928-11-05 (lundi)}
\label{rec:Paris4}

Ce récital a eu lieu à~21~heures, à la salle \Chopin{} de la maison
\Pleyel{}, rue Daru, \Number{8}.
Son programme était le suivant \citep[voir][]{Nekrasova08, Semaine334b,
Semaine335, White}~:
\begin{itemize}
 \item
 \Chopin{}~: Fantaisie en \kF mineur, \Opus{49}~; Ballades en \kG mineur et
 en \kF mineur, \Opus{23} et \Opus{52}~; Huit Préludes~; Sonate en \kB
 mineur, \Opus{58}~; Nocturne en \kF majeur, \Opus{15} \Number{1}~; Quatre
 Études~; Barcarolle en \kF \Sharp majeur, \Opus{60}~; Tarentelle en \kA
 \Flat majeur, \Opus{43}~; Deux Mazurkas~; Scherzo en \kB mineur, \Opus{20}.
\end{itemize}

\section{Cinquième récital, 1929-01-11 (vendredi)}
\label{rec:Paris5}

Ce récital a eu lieu à~21~heures, à la salle \Chopin{} de la maison
\Pleyel{}, rue Daru, \Number{8}.
Son programme était le suivant \citep[voir][]{Heugel4836, Nekrasova08,
Semaine346, White}~:
\begin{itemize}
 \item
 \Schumann{}~: \emph{Bunte Blätter}, \Opus{99}~; Deux \emph{Novelettes},
 \Opus{21} (\Number{7} en \kE majeur et \Number{8} en \kF \Sharp mineur)~;
 \item
 \Liszt{}~: Après une lecture de Dante, S~161 \Number{7}~;
 \item
 \Ravel{}~: Sonatine~;
 \item
 \Debussy{}~: \emph{Doctor Gradus ad Parnassum}, L~113 \Number{I}~;
 \emph{General Lavine -- eccentric}, L~123 \Number{VI}~;
 \item
 \Prokofiev{}~: Cinq Sarcasmes, \Opus{17}~;
 \item
 \Scriabine{}~: Poème en \kF \Sharp majeur, \Opus{32} \Number{1}~; Étude en
 \kD \Sharp mineur, \Opus{8} \Number{12}~; Sonate \Number{4} en \kF \Sharp
 majeur, \Opus{30}.
\end{itemize}

Ce récital du~11 janvier~1929 fit l'objet d'un compte-rendu dans \emph{Le
Ménestrel} \citep[voir][]{Baruzi}.

\begin{quotation}
 Je ne sais rien jusqu'ici de M.~Wladimir Sofronitzky.
 Rien de ce qui pour lui se passa avant sa venue parmi nous.
 Nul détail biographique~; nul fait particulier.
 Mais mieux que ne le ferait tout détail, m'éclairent, me
 \hbox{semble-t-il}, la netteté des traits de ce visage et la méditative, --
 parfois hautainement triste, -- franchise de ce regard~; -- en ce jeu cette
 totale absence d'emphase et de flatterie et, dès la première note frappée,
 \emph{cette sorte de puissance d'exil} (comme si, pour celui qui commence
 d'interpréter ces œuvres, cette salle soudain n'existait plus, mais
 uniquement ces œuvres elles-mêmes, et cet instrument où s'achève leur être.
 De toute la vérité qui est en lui, -- et qui est identique à son plus haut
 rêve, -- ce pianiste, désormais, et jusqu'à l'instant où se déploiera
 l'accord final, sera tout entier en ce qu'il fait revivre).

 Que m'importent, dès lors, çà et là, tels moments où la technique déjà
 puissante paraît comme refuser de se livrer tout entière~; ou tels autres
 encore, où les plus complexes intentions se font rétractiles, presque en
 une méfiance de leur propre force~?
 Celui qui a joué de si profonde manière la \emph{Fantasia quasi Sonata}, en
 donnant l'impression de n'avoir point uniquement scruté et pénétré le texte
 de Liszt, mais à un non moindre degré celui de Dante, et jusqu'à avoir,
 comme Liszt lui-même, écouté au plus intime de sa pensée le dialogue de
 Dante et de Virgile parmi les rafales de l'\emph{Inferno}, -- celui-là, dès
 maintenant, laisse très loin derrière lui tous les virtuoses qui ne sont
 que virtuoses, si intense et démesuré que puisse aller vers eux le succès.
 Et cette profondeur et cette acuité qui en de tels instants animaient son
 jeu, combien les retrouvai-je plus tard, à l'occasion de maintes autres
 œuvres, le \emph{Cinquième Sarcasme} de Prokofieff, et le \emph{Poème},
 op.~32, l'\emph{Étude en \kE \Flat} [sic], la \emph{Quatrième Sonate} de
 Scriabine~!%
 \sourceatright{\citep{Baruzi}}
\end{quotation}

\section{Sixième récital, 1929-05-17 (vendredi)}
\label{rec:Paris6}

Ce récital a eu lieu à~21~heures, à la salle \Chopin{} de la maison
\Pleyel{}, rue Daru, \Number{8}.
Son programme était le suivant \citep[voir][]{Heugel4854, Nekrasova08,
Semaine363, Semaine364, White}~:
\begin{itemize}
 \item
 \Beethoven{}~: Sonate \Number{32} en \kC mineur, \Opus{111}~;
 \item
 \Schumann{}~: Carnaval, \Opus{9}~;
 \item
 \Chopin{}~: Nocturne en \kF \Sharp majeur, \Opus{15} \Number{2}~; Mazurka
 en \kC \Sharp mineur~; Scherzo en \kB mineur, \Opus{20}~;
 \item
 \Liszt{}~: Valse oubliée~;
 \item
 \Schubert{}/\Liszt{}~: Valse-caprice~; \emph{Erlkönig}, S~558 \Number{4}.
\end{itemize}

Ce récital du~17 mai~1929 fit l'objet d'un compte-rendu dans \emph{L'Œuvre}
\citep[voir][]{oeuvre1929-05-24}.

\begin{quotation}
 Le pianiste Sofronitzky a joué vendredi dernier~[1929-05-17] le Carnaval de
 \Schumann{}, la Valse en \kA de \Schubert{}/\Liszt{}, des Nocturnes de
 \Chopin{} et la Sonate op.\@~111 de \Beethoven{}, avec la sensibilité
 et~[le] goût les plus parfaits.
 La technique achevée de ce virtuose lui permet de donner sa véritable
 valeur à chaque œuvre.
 Son style personnel imprégné de la plus agréable musicalité, ses sonorités
 veloutées ont fait la juste admiration d'un nombreux auditoire.%
 \sourceatright{\citep{oeuvre1929-05-24}}
\end{quotation}

\section{Septième récital, 1929-12-20 (vendredi)}
\label{rec:Paris7}

Ce récital a eu lieu à~21~heures, à la salle \Chopin{} de la maison
\Pleyel{}, rue Daru, \Number{8}.
Selon \citet[p.~401]{Scriabine}, il s'agit plutôt de la salle \Debussy{}~;
le fils du pianiste indique la salle \Chopin{}.
Son programme était le suivant \citep[voir][]{Nekrasova08, Semaine392,
Semaine394, White}~:
\begin{itemize}
 \item
 \Chopin{}~: Deux Sonates pour piano (\Number{3} en \kB mineur, \Opus{58}~;
 \Number{2} en \kB \Flat mineur, \Opus{35})~;
 \item
 \Scriabine{}~: Sonate pour piano \Number{10}, \Opus{70}~; Sonate pour piano
 \Number{5}, \Opus{53}~; autres œuvres de forme brève.
\end{itemize}

\section*{Remarque}

À propos du troisième récital~[1928-10-29] et du quatrième~[1928-11-05], le
journal \emph{La Liberté} \citep[voir][]{liberte1928-10-26} évoque en outre
la radio\-diffusion de leur émission \emph{Radio Liberté}, le vendredi~26
octobre~1928 à~12\up{h}\,30\up{m}, lors de laquelle \VSofronitsky{} a joué
trois Préludes ainsi que la Barcarolle en \kF \Sharp majeur, \Opus{60}, de
\Chopin{}.
